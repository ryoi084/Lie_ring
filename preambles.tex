\usepackage{amsmath,amssymb,amsthm}
\usepackage{pifont}
\usepackage[top=2cm, bottom=2cm, left=1cm, right=1cm]{geometry}
\usepackage{physics}
\usepackage{mathtools}
\usepackage{thmtools}
\usepackage{mdframed}
\usepackage[usename,dvipsnames]{xcolor}
\usepackage{hyperref}
\usepackage{enumitem}
\usepackage[nameinlink,capitalize]{cleveref}
\usepackage{titlesec}
\usepackage[deluxe]{otf}
\usepackage{tikz}
\usetikzlibrary{intersections, calc, arrows.meta}
\titleformat*{\section}{\centering\LARGE\bfseries\rmfamily}
\titleformat*{\subsection}{\large\bfseries\rmfamily}
\titlespacing*{\subsection}{0pt}{1cm}{0pt}
\titleformat*{\subsubsection}{\bfseries\rmfamily}
\allowdisplaybreaks[4]
\definecolor{dark-blue}{cmyk}{1,0.2,0,0.6}
\hypersetup{
    colorlinks=true,
    linkcolor=dark-blue,
}

\numberwithin{equation}{section}
\declaretheoremstyle[
headfont=\normalfont\bfseries,
notefont=\bfseries, notebraces={『}{』},
bodyfont=\normalfont,
postheadspace=\newline,
headpunct={},
within=section,
preheadhook={\begin{mdframed}[backgroundcolor=orange!15,
  innertopmargin =0pt , splittopskip = \topskip,
  skipbelow= 6pt, skipabove=6pt, %
  topline=false,bottomline=false,leftline=false,rightline=false]},
postfoothook=\end{mdframed}
]{defstyle}


\declaretheorem[
  name=Definition,
  style=defstyle,
  ]{definition}
\declaretheorem[
  name=Theorem,
  within=section,
  thmbox={
    style=M,
    % titlestyle={『{}』},
    bodystyle=\normalfont\noindent}
  ]{theorem}
\declaretheorem[
  name=Lemma,
  within=section,
  thmbox={style=M,bodystyle=\normalfont}
  ]{lemma}
\declaretheorem[
  name=Probrem,
  within=section,
  thmbox={style=S,bodystyle=\normalfont}
  ]{problem}
\declaretheorem[
  name={\rm \textbf{Solution}},
  style=remark,
  within=section,
  ]{solution}
\declaretheorem[
  name=Remark,
  style=remark,
  numbered=no,
  ]{remark}
\declaretheorem[
  name=\textbf{Named},
  style=remark,
  shaded={bgcolor=gray!15},
  numbered=no,
  ]{named}
\declaretheoremstyle[
  spaceabove=6pt, spacebelow=6pt,
  headfont=\normalfont\bfseries,
  notefont=\mdseries,
  bodyfont=\normalfont,
  postheadspace=1em,
  ]{exstyle}
\declaretheorem[sharenumber=theorem,style=exstyle]{example}
\declaretheoremstyle[
    spaceabove=-6pt, 
    spacebelow=6pt, 
    headfont=\normalfont\bfseries, 
    bodyfont = \normalfont,
    postheadspace=1em, 
    qed=$\blacksquare$, 
    headpunct={:}]{myproofstyle} %<---- change this name
\declaretheorem[name={Proof}, style=myproofstyle, unnumbered]{Proof}

\DeclareMathOperator{\Der}{Der}
\DeclareMathOperator{\Sym}{Sym}
\DeclareMathOperator{\Alt}{Alt}
\DeclareMathOperator{\Ker}{Ker}
\DeclareMathOperator{\ad}{ad}
% \DeclareMathOperator{\o}{o}
\newcommand{\dprime}{{\prime\prime}}
\renewcommand{\o}{o}
\newcommand{\R}{\mathbb{R}}
\newcommand{\g}{\mathfrak{g}}
\newcommand{\h}{\mathfrak{h}}
\renewcommand{\a}{\mathfrak{a}}
\renewcommand{\b}{\mathfrak{b}}
% \newcommand{\gl}{g\ell}
\newcommand{\ie}{\textit{i}.\textit{e}.\gt }
\newcommand{\gl}{gl}
\renewcommand{\sl}{sl}
\newcommand{\cmark}{\ding{51}}
\newcommand{\xmark}{\ding{55}}
\newcommand{\htar}[2]{\hypertarget{#1}{#2}}
\newcommand{\hlin}[2]{\hyperlink{#1}{#2}}
\renewcommand{\href}[2]{\hyperref[#1]{#2}}
\renewcommand{\emph}[1]{\textbf{#1}}
\renewcommand\labelitemi{$\circ$}
\crefname{theorem}{Thm.}{Thms.}
\crefname{definition}{Def.}{Defs.}
\crefname{lemma}{Lem.}{Lems.}

\renewenvironment{abstract}
 {\footnotesize
  \begin{center}
  \bfseries \vspace{-.5cm}\vspace{0pt}
  \end{center}
  \list{}{
    \setlength{\leftmargin}{3.5cm}%
    \setlength{\rightmargin}{\leftmargin}%
    \setlength{\partopsep}{.5cm}%
  }%
  \item\relax}
 {\endlist}

\newcommand{\環}          {\href{Def:1_1}{環}}
\newcommand{\代数}        {\href{Def:1_2}{代数}}
\newcommand{\Lie環}       {\href{Def:1_3}{リー環}}
\newcommand{\部分Lie環}    {\href{Def:2_1}{部分リー環}}
\newcommand{\線形部分空間}  {\href{Def:2_1_i}{線形部分空間}}
\newcommand{\イデアル}     {\href{Def:2_2}{イデアル}}
\newcommand{\準同型写像}    {\href{Def:2_3}{準同型写像}}
\newcommand{\表現}        {\href{Def:rep}{表現}}
\newcommand{\同値関係}     {\href{Def:equiv_rel}{同値関係}}
\newcommand{\交代積}       {\href{Def:comm}{交代積}}
\newcommand{\微分}        {\href{Def:diff}{微分}}
\newcommand{\商環}        {\href{Def:quot_ring}{商環}}
\newcommand{\クラス}       {\href{Def:class}{クラス}}
\newcommand{\Lie環の同型}  {\href{Def:isomorphic}{リー環の同型}}
\newcommand{\標準的準同型写像}{\href{Def:canon_hom}{標準的準同型写像}}
\newcommand{\可換なLie環}  {\href{thm:comm_lie_ring}{可換なリー環}}
\newcommand{\実数体}       {\href{nam:R_F}{実数体}}