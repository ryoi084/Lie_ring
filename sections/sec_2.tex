\documentclass[../main]{subfiles}
\usepackage{../settings/myexternaldocument}
\setcounter{section}{1}

\begin{document}

\twocolumn[
  \section{部分環、イデアル、準同形}\label{sec:2}
  \begin{abstract}
    この章ではリー環論を進める上で基礎になるいくつかの概念を説明する。これらは他の代数系(群、結合環、etc.)についても同様に定義されるもので、いわば「抽象代数」全体に共通するものである。
  \end{abstract}
]

\subsection{部分環}
$\g$を実数体$\R$上の{\Lie環}とする

\begin{definition}[部分リー環]\label{Def:2_1}
  $\g$の部分集合$\h$が次の二つの条件を満たすとき、$\h$は$\g$の\emph{部分(リー)環}((Lie) subalgebra)であるという。
  \begin{enumerate}[label=(\roman*)]
    \item \label{Def:2_1_i}$\h$は$\g$の{\線形部分空間}である。すなわち
      \begin{align*}
        x,y\in\h,\;\lambda\in\R~\Rightarrow~ x+y\in\h,\;\lambda x\in\h.
      \end{align*}
    \item \label{Def:2_1_ii}$x,y\in\h\Rightarrow\comm{x}{y}\in\h.$
  \end{enumerate}
\end{definition}

\begin{remark}
  $\h$は$\g$の演算(加法、乗法、スカラー乗法)に関して閉じているため、$\h$もまた一つの{\環}({\代数})になる。
\end{remark}
\begin{remark}
  {\Lie環}の条件\ref{Def:1_3_I}\ref{Def:1_3_II}も、$\g$全体で成立しているから、もちろん$\h$の中でも成立する。したがって、$\h$はそれ自身一つのリー環になる。
\end{remark}

\subsubsection*{Example 1.}
\begin{remark}
  \hlin{nam:gl}{$\g=\gl_n(\R)$}とし
  \begin{align}
    \h=\qty{x\in\gl_n(\R) \mid \tr(x)=0} \label{eq:sl}
  \end{align}
  とおく。すなわち$x=(\xi_{ij})$に対して$\tr(x)=\sum_{i=1}^{n}\xi_{ii}$は1次方程式$\tr(x)=0$の解空間であるから、$\h$は$\g$の{\線形部分空間}である。

  また$x,y\in\h$、$x=\qty(\xi_{ij})$、$y=\qty(\eta_{ij})$に対し
  \begin{align*}
    \tr(xy) =& \sum_{i,j=1}^{n}\xi_{ij}\eta_{ji},\\
    \tr(yx) =& \sum_{i,j=1}^{n}\eta_{ij}\xi_{ji}
  \end{align*}
  であるから、$\tr(xy)=\tr(yx)$。したがって
  \begin{align*}
    \tr(\comm{x}{y})=&\tr(xy-yx)\\
    =&\tr(xy)-\tr(yx)=0 \tag{3}\label{eq:2_(3)}
  \end{align*}
  となり、$\comm{x}{y}\in\h$。よって$\h$は$\g$の{\部分Lie環}である。
\end{remark}

\begin{named}
  \eqref{eq:sl}で表される$\h$を\htar{nam:sl}{$\sl_n(\R)$}または$\sl(n,\R)$と書く。($\sl$はspecial linearの略。)
\end{named}

\begin{remark}
  この場合、実際には元に対して\eqref{eq:sl}の制限は必要ではなく、任意の$x,y\in\g$に対し$\comm{x}{y}\in\h$が成立する。
\end{remark}


\subsubsection*{Example 2.}
\begin{definition}[実対称$\cdot$実交代行列]\label{Def:Sym_and_Alt}
  $n$次\htar{Def:Sym}{実対称行列}の全体を以下のように置く。
  \begin{align*}
    \Sym_{n}(\R) =& \qty{x\in\gl_{n}(\R)\mid ^{t}x=x}.
  \end{align*}
  $n$次\htar{Def:Alt}{実交代行列}の全体を以下のように置く。
  \begin{align*}
    \Alt_{n}(\R) =& \qty{x\in\gl_{n}(\R)\mid ^{t}x=-x}.
  \end{align*}
  ($t$は転置行列を表す)
\end{definition}
\begin{remark}
  \cref{Def:Sym_and_Alt}は明らかに$\gl_{n}(\R)$の{\線形部分空間}になる。
\end{remark}
\begin{theorem}
  $\Alt_{n}(\R)$は{\部分Lie環}である。$\Sym_{n}(\R)$の方は\underline{部分リー環にはならない。}
\end{theorem}
\begin{Proof}
  一般に$x,y\in\gl_{n}(\R)$に対し
  \begin{align*}
    ^{t}\comm{x}{y} =& ^{t}\qty(xy-yx)\\
    =& ^{t}y^{t}x-^t{x}^t{y}\\
    =& -\comm{^{t}x}{^{t}y}
  \end{align*}
  であるから、
  \begin{align*}
    \begin{cases}
      x,y\in\Sym_{n}(\R) &\Rightarrow \comm{x}{y}\in\Alt_{n}(\R)\\
      x\in\Sym_{n}(\R),y\in\Alt_{n}(\R) &\Rightarrow \comm{x}{y}\in\Sym_{n}(\R)\\
      x,y\in\Alt_{n}(\R) &\Rightarrow \comm{x}{y}\in\Alt_{n}(\R)
    \end{cases}
    \tag{4}\label{eq:2_(4)}
  \end{align*}
  よって$\Alt_{n}(\R)$のみ{\部分Lie環}となる。
\end{Proof}
\begin{named}
  $\Alt_{n}(\R)$を\htar{nam:o}{$\o_{n}(\R)$}または$\o(n)$ともかく。
\end{named}
\begin{remark}
  \begin{align*}
    \hlin{nam:o}{\o(n)}\subset \hlin{nam:sl}{\sl_{n}(\R)} \subset \hlin{nam:gl}{\gl_{n}(\R)} \tag{5}\label{eq:2_(5)}
  \end{align*}
  であるから、$\o(n)$はまた$\sl_{n}(\R)$の{\部分Lie環}でもある。($\o(n)=\Alt_{n}(\R)$は自然に\eqref{eq:sl}の条件を満たしている。)
\end{remark}


\subsubsection*{Example 3.}
\begin{theorem}\label{thm:gl}
  \begin{enumerate}[label=(\alph*)]
    \item \label{thm:2_ex3_a}$A$を単に($\R$上の)\hlin{vec_space}{ベクトル空間}とみなし、その1次変換($A$から$A$の中への一次写像)全体の集合を$\gl(A)$と書くことにすれば、これも一つの($\R$上の){\Lie環}である。
    \item \label{thm:2_ex3_b}\href{Def:der_alg}{$\Der(A)$}は$\gl(A)$の{\部分Lie環}になる。
  \end{enumerate}
\end{theorem}

\begin{remark}
  \ref{thm:2_ex3_a}について、$\dim A=n$とし$A$の$\R$上の一つの底をとって考えれば、$\gl(A)$は\hlin{nam:gl}{$\gl_{n}(\R)$}と同じものである。
\end{remark}

\begin{Proof}
  \ref{thm:2_ex3_b}について$A$が\hlin{nam:trivial_alg}{トリビアルな代数}ならば、$x,y\in A$、$D\in\href{Def:der_alg}{\Der(A)}$として{\微分}の条件\eqref{eq:1_(9)}を確認すると
  \begin{align*}
    &D(xy) = D(x)\cdot y + x\cdot D(y)& \\
    \Leftrightarrow\;& D(0) = D(x)\cdot y + x\cdot D(y) &\;&\because xy=0\qq{from \eqref{eq:1_(4)}} \\
    \Leftrightarrow\;& 0 = 0+0 &\;&\because D(x),D(y)\in A
  \end{align*}
  となり常に成立する。よって$\Der(A)=\gl(A)$となる。
\end{Proof}

\subsection{イデアル}

\begin{definition}[イデアル]\label{Def:2_2}
  $\g$の部分集合$\h$が次の2つの条件を満たすとき、$\h$は$\g$の\emph{イデアル}(ideal)であるという。
  \begin{enumerate}[label=(\roman*)$^\prime$]
    \item \label{Def:2_2_i'}$\h$は$\g$の{\線形部分空間}である。
    \item \label{Def:2_2_ii'}すべての$x\in\g,y\in\h$に対し、$\comm{x}{y}\in\h$。
  \end{enumerate}
\end{definition}

\begin{remark}
  明らかに\ref{Def:2_2_ii'}$\Rightarrow$\ref{Def:2_1_ii}である。したがって{\イデアル}はすべて{\部分Lie環}だが、逆は必ずしも成立しない。
\end{remark}
\begin{theorem}\label{thm:2_ideal}
  \begin{enumerate}[label=(\alph*)]
    \item \label{thm:2_ideal_a}\hlin{nam:sl}{$\sl_n(R)$}は\hlin{nam:gl}{$\gl_n(R)$}のイデアル。
    \item \label{thm:2_ideal_b}\hlin{nam:o}{$\o(n)$}は\hlin{nam:gl}{$\gl_n(R)$}の(部分リー環であるが)イデアルではない
  \end{enumerate}
\end{theorem}

\begin{Proof} \cref{thm:2_ideal}\ref{thm:2_ideal_a}については、\cref{sec:1}\eqref{eq:1_(3)}より自明。
\end{Proof}

\begin{Proof} \cref{thm:2_ideal}\ref{thm:2_ideal_b}について、$x\in\hlin{Def:Sym}{\Sym_n(\R)}\subset\gl_n(\R)$、$y\in\hlin{Def:Alt}{\Alt_n(\R)}$とすると($\Sym_n(\R)$は$\gl_n(\R)$の{\部分Lie環}ではないが部分集合ではある。)、\eqref{eq:2_(4)}の2式より
  \begin{align*}
    &\comm{x}{y}\in\Sym_n(\R)\qc
    \comm{x}{y}\notin\Alt_n(\R)\\
    &\therefore \comm{x}{y}\notin\o(n)
  \end{align*}
  となるため、\ref{Def:2_2_ii'}が成立しない。
\end{Proof}

\begin{remark}
  リー環論では(つねに\eqref{eq:1_(5')}より$\comm{x}{y}=-\comm{y}{x}$であるため)"右{\イデアル}"と"左イデアル"は区別の必要がない。
\end{remark}

\begin{named}
  $\g$の{\線形部分空間}\htar{nam:[a,b]}{$\a,\b$}に対し$\comm{x}{y}(x\in\a,y\in\b)$によって生成される部分空間を$\comm{\a}{\b}$で表す。したがって$\comm{\a}{\b}$は
  \begin{align*}
    \sum_{i=1}^{m}\comm{x_{i}}{y_{i}}\qc x_{i}\in\a,y_{i}\in\b,m\ge1
  \end{align*}
  の形の有限和全体の集合である。\textcolor{red}{$\comm{x_{i}}{y_{j}}$ではない?}
\end{named}
\begin{remark}
  $\comm{x}{y}$の形の元全体の集合は必ずしも{\線形部分空間}にはならない。
\end{remark}

\begin{lemma}\label{lem:2_1}
  \hlin{nam:[a,b]}{$\a,\b$}が$\g$の{\イデアル}ならば、$\comm{\a}{\b}$も$\g$のイデアルである。
\end{lemma}

\begin{Proof}
  $x\in\g,y\in\a,z\in\b$のとき、$\comm{x}{\comm{y}{z}}$が$\comm{\a}{\b}$に含まれていることをいえばよい。\eqref{eq:1_(5)}、\eqref{eq:1_(6)}により
  \begin{align*}
    \mathrm{\eqref{eq:1_(6)}} 
    \Leftrightarrow
    & \comm{\comm{x}{y}}{z}-\comm{x}{\comm{y}{z}}-\comm{y}{-\comm{x}{z}}=0\\
    \Leftrightarrow
    & \comm{x}{\comm{y}{z}}=\comm{\comm{x}{y}}{z}+\comm{y}{\comm{x}{z}} \tag{6}\label{eq:2_(6)}
  \end{align*}
  が成立する。$\a$が$\g$の{\イデアル}だから、$\comm{x}{y}\in\a$、よって$\comm{\comm{x}{y}}{z}\in\comm{\a}{\b}$。また、$\b$が$\g$のイデアルだから、$\comm{x}{z}\in\b$、よって$\comm{y}{\comm{x}{z}}\in\comm{\a}{\b}$。よって$\comm{x}{\comm{y}{z}}\in\comm{\a}{\b}$。
\end{Proof}

\begin{definition}[導イデアル]\label{Def:der_ideal}
  \cref{lem:2_1}により、$\g$自身$\g$の{\イデアル}だから、$\comm{\g}{\g}$もイデアルである。これを$D\g$とかき、$\g$の\emph{導イデアル}(derived ideal)という。
\end{definition}
\begin{remark}
  \cref{Def:der_ideal}よりさらに
  \begin{align*}
    D^{2}\g =& D(D\g) = \comm{\comm{\g}{\g}}{\comm{\g}{\g}}\\
    \g^{3} =& \comm{\g}{D\g} = \comm{\g}{\comm{\g}{\g}},\cdots
  \end{align*}
  等の{\イデアル}が作られる。
\end{remark}

\begin{problem}\label{prob:1}
  $\g=\hlin{nam:gl}{\gl_n(\R)}$のとき、$\href{Def:der_ideal}{D\g}=\hlin{nam:sl}{\sl_n(\R)}$となることを示せ。
  \href{sol:1}{(回答)}
\end{problem}

\subsection{準同型}
$\g$、$\g^\prime$を($\R$上の){\Lie環}とする。

\begin{definition}[準同型写像]\label{Def:2_3}
  $\g$から$\g^\prime$の中への写像$\varphi$が次の条件を満たすとき、$\varphi$は({\Lie環}の)\emph{準同型写像}(homomorphism)という。
  \begin{enumerate}[label=(\roman*)]
    \item \label{Def:2_3_i}$\varphi$は$\g$から$\g^\prime$への線形写像である。
    \item \label{Def:2_3_ii}任意の$x,y\in\g$に対し
      \begin{align*}
        \varphi\qty(\comm{x}{y}) = \comm{\varphi(x)}{\varphi(y)}\tag{7}\label{eq:2_(7)}
      \end{align*}
      が成立する。
  \end{enumerate}
  つまり、リー環の準同型写像とはリー環の演算(加法、乗法、スカラー乗法)を保存する写像のことである。
\end{definition}

\begin{theorem}[合成写像]
  $\g^{\dprime}$を第三の{\Lie環}、$\varphi^\prime$を$\g^\prime$から$\g^{\dprime}$の中への{\準同型写像}とすれば、合成写像$\varphi^\prime\circ\varphi\colon\g\longrightarrow\g^{\dprime}$も準同型写像になる。
\end{theorem}

\begin{Proof}
  \ref{Def:2_3_ii}の条件を調べてみると
  \begin{align*}
    \qty(\varphi^{\prime}\circ\varphi)\qty(\comm{x}{y}) =& \varphi^{\prime}\qty(\varphi\qty(\comm{x}{y}))\\
    =& \varphi^{\prime}\qty(\comm{\varphi(x)}{\varphi(y)})\\
    =& \comm{\varphi^{\prime}\qty(\varphi(x))}{\varphi^{\prime}(\varphi(y))}\\
    =& \comm{\varphi^{\prime}\circ\varphi(x)}{\varphi^{\prime}\circ\varphi(y)}
  \end{align*}
  から確かに成立する。同様に$\varphi^{\prime}\circ\varphi$は\ref{Def:2_3_i}も満たす。
\end{Proof}

\begin{theorem}[逆写像の準同型写像]
  {\準同型写像}$\varphi\colon\g\longrightarrow\g^{\prime}$が1対1(単射)かつ$\g^{\prime}$の上への写像(全射)であるときには$\varphi$の逆写像$\varphi^{-1}$が存在する。そのとき、$\varphi^{-1}$は$\g^{\prime}$から$\g$(の上)への準同型写像になる。
\end{theorem}

\begin{Proof}
  条件\ref{Def:2_3_ii}を調べるために$x^{\prime},y^{\prime}\in\g^{\prime}$をとり$x=\varphi^{-1}(x^{\prime}),y=\varphi^{-1}(y^{\prime})$とすれば、$\varphi(x)=x^{\prime},\varphi(y)=y^{\prime}$。よって
  \begin{align*}
    \varphi\qty(\comm{x}{y})=\comm{\varphi(x)}{\varphi(y)}=\comm{x^{\prime}}{y^{\prime}}.
  \end{align*}
  であり、両辺を$\varphi^{-1}$で移すと(都合で右辺と左辺を入れ替える)
  \begin{align*}
    \varphi^{-1}\comm{x^{\prime}}{y^{\prime}} =& \qty(\varphi^{-1}\circ\varphi)\qty(\comm{x}{y}) \\
    =& \comm{x}{y} \\
    =& \comm{\varphi^{-1}(x^{\prime})}{\varphi^{-1}(y^{\prime})}
  \end{align*}
  となるため、\ref{Def:2_3_ii}が成立する。同様に$\varphi^{-1}$は\ref{Def:2_3_i}も満たす。
\end{Proof}

\begin{named}\htar{nam:isomorphism}{}
  1対1、上への写像であるような{\準同型写像}$\varphi\colon\g\longrightarrow\g^{\prime}$を({\Lie環}の)\emph{同型写像}(isomorphism)という。
\end{named}

\begin{remark}
  上述のように$\varphi$が\hlin{nam:isomorphism}{同型写像}なら$\varphi^{-1}\colon\g^{\prime}\longrightarrow\g$も同型写像である。また$\varphi^{\prime}\colon\g^{\prime}\longrightarrow\g^{\dprime}$も同型写像ならば、明らかに$\varphi^{\prime}\circ\varphi$も同型写像である。
\end{remark}

\begin{definition}[リー環の同型]\label{Def:isomorphic}
  二つの{\Lie環}$\g,\g^{\prime}$の間に\hlin{nam:isomorphism}{同型写像}$\varphi\colon\g\longrightarrow\g^{\prime}$が存在するとき、$\g$と$\g^{\prime}$は\emph{同型}(isomorphic)であるといい、$\g\cong\g^{\prime}$、$\g \stackrel{\sim}{\longrightarrow}\g^{\prime}$とかく。
\end{definition}

\begin{remark}
  \cref{Def:isomorphic}は{\Lie環}の集合に一つの{\同値関係}を定義する。
\end{remark}

\begin{definition}[核と像]\label{Def:ker_im}
  {\準同型写像}$\varphi\colon\g\longrightarrow\g^{\prime}$が与えられたとき
  \begin{align*}
    \Ker\varphi =& \qty{x\in\g\mid\varphi(x)=0} \\
    \Im\varphi =& \qty{\varphi(x)\mid x\in\g}
  \end{align*}
  とおき、それぞれ$\varphi$の\htar{Def:ker}{\emph{核}}(kernel)、\htar{Def:im}{\emph{像}}(image)という。
\end{definition}

\begin{remark}
  $\Ker\varphi=\varphi^{-1}(0)$、$\Im\varphi=\varphi(\g)$ともかく。
\end{remark}

\begin{theorem}\label{thm:ker_im}
  $\hlin{Def:ker}{\Ker}\varphi$は$\g$の{\イデアル}、$\hlin{Def:im}{\Im}\varphi$は$\g^{\prime}$の{\部分Lie環}になる。
\end{theorem}

\begin{Proof}
  \begin{itemize}
    \item \hlin{Def:ker}{$\Ker\varphi$}については、
      $x\in\g,y\in\Ker\varphi$とするとき、条件\ref{Def:2_2_i'}は明らか。\ref{Def:2_2_ii'}について
      \begin{align*}
        \varphi\qty(\comm{x}{y}) = \comm{\varphi(x)}{\varphi(y)}=\comm{\varphi}{0}=0
      \end{align*}
      であるから、$\comm{x}{y}\in\Ker\varphi$となり成立する。よって$\Ker\varphi$は$\g$の{\イデアル}である。
    \item \hlin{Def:im}{$\Im\varphi$}については、$x,y\in\Im\varphi$とするとき、条件\ref{Def:2_1_i}は明らか。\ref{Def:2_1_ii}について、$\varphi(x),\varphi(y)\in\g^{\prime}$とおくと
      \begin{align*}
        \comm{\varphi(x)}{\varphi(y)} = \varphi\qty(\comm{x}{y})
      \end{align*}
      となるが、$\comm{x}{y}\in\g$であるので$\varphi\qty(\comm{x}{y})\in\g^{\prime}$が成立する。よって、$\Im\varphi$は$\g^{\prime}$の{\部分Lie環}である。
  \end{itemize}
\end{Proof}

\subsubsection*{Example 4.}
\begin{definition}[表現]\label{Def:rep}
  一般に{\Lie環}$\g$からある\hlin{vec_space}{ベクトル空間}$V$の一次変換の作るリー環$\gl(V)$の中への{\準同型写像}$\rho$を$\g$の$V$における"\emph{表現}"(representation)という。
\end{definition}

\begin{definition}[随伴表現]\label{Def:adj_rep}
  $\g$を任意の{\Lie環}とし、$x\in\g$の左乗によって定義される$\g$の一次変換
  \begin{align*}
    y\mapsto\comm{x}{y}\qq{}(y\in\g)
  \end{align*}
  を$\ad(x)$とかく。これは$\g$の$\g$自身における{\表現}である。これを$\g$の\emph{随伴表現}(adjoint representation)という。
\end{definition}

\begin{remark}
  具体的には$\qty(\ad(x))(y)=\comm{x}{y}$。
\end{remark}

\begin{theorem}
  \href{Def:adj_rep}{$\ad(x)$}は\href{thm:2_ex3_b}{$\gl(\g)$}の元であり、$\ad(x)\in\href{Def:der_alg}{\Der(\g)}$である。
\end{theorem}

\begin{Proof}
  $\ad(x)$が$\gl(\g)$の元であることは\cref{thm:gl}\ref{thm:2_ex3_a}より明らか。また
  \begin{align*}
    (\ad(x))\qty(\comm{y}{z}) =& \comm{x}{\comm{y}{z}}\\
    =& \comm{\comm{x}{y}}{z}+\comm{y}{\comm{x}{z}} \qq{} \because \qq*{from \eqref{eq:2_(6)}}\\
    =& \comm{\qty(\ad(x))(y)}{z}+\comm{y}{\qty(\ad(x))(z)}
  \end{align*}
  となるため、{\微分}の条件\eqref{eq:1_(9)}が成立している。よって$\ad(x)\in\Der(\g)$
\end{Proof}

\subsection{商環と標準的準同型}
\begin{theorem}[同値関係]\label{thm:equiv_rel}
  $\g$を{\Lie環}、$\a$をその{\イデアル}とする。$x,y\in\g$に対し、合同関係を
  \begin{align*}
    x\equiv y \pmod{\a} \Leftrightarrow x-y\in\a
    \tag{9}\label{eq:2_(9)}
  \end{align*}
  によって定義すると、これは一つの同値関係になっている。すなわち、次の三つの条件\ref{Def:equiv_rel_1)},\ref{Def:equiv_rel_2)},\ref{Def:equiv_rel_3)}が成立する。
  \begin{enumerate}[label=\arabic*)]
    \item \label{Def:equiv_rel_1)} すべての$x\in\g$に対し$x\equiv x \pmod{\a}$.
    \item \label{Def:equiv_rel_2)}$x\equiv y \pmod{\a}$ $\Leftrightarrow$ $y\equiv x \pmod{\a}$.
    \item \label{Def:equiv_rel_3)}$x\equiv y$, $y\equiv z\pmod{\a}$ $\Leftrightarrow$ $x\equiv z \pmod{\a}$.
  \end{enumerate}
\end{theorem}

\begin{Proof}
  \ref{Def:equiv_rel_1)}$x-x=0\in\a$だから。\ref{Def:equiv_rel_2)}仮定により$x-y\in\a$だから。\ref{Def:equiv_rel_3)}仮定により$x-y\in\a,y-z\in\a$。よって$x-z=(x-y)+(y-z)\in\a$となるから。
\end{Proof}

\begin{remark}
  \eqref{eq:2_(9)}の合同が{\同値関係}になることは、上の証明より$\a$が加法群になっていることからの結果である。
\end{remark}

\begin{definition}[類]\label{Def:class}
  {\同値関係}に関する同値類の集合を$\g/\a$とかく。$x\in\g$を含む"同値類"とは$x$と合同な元全体の集合
  \begin{align*}
    X=&\qty{x^{\prime}\in\g \mid x\equiv x^{\prime} \pmod{\a}} \\
    =& \qty{x+a\mid a\in\a}
  \end{align*}
  である。これを以下$x$の\emph{類}(クラス)といい、$\bar{x}$または$x+\a$で表すことにする。
\end{definition}


\begin{theorem}
  同値類の集合$\g/\a$には自然にまた和、スカラー倍、\hlin{Lie_bra}{リー積}が定義できる。
\end{theorem}

\begin{Proof}
  \begin{itemize}
    \item 和は
      \begin{align*}
        X=\bar{x}\qc Y=\bar{y} \qq{のとき} X+Y=\overline{x+y}
        \tag{10}\label{eq:2_(10)}
      \end{align*}
      と定義するのが自然であろう、すなわち、$X,Y\in\g/\a$, $x\in X$,$y\in Y$であるとき、$X+Y$を$x+y$を含む{\クラス}として定義しようというのである。ここで大切なことは$X,Y$の"代表"$x,y$のとり方は幾通りもあり、したがって$x+y$は色々であるが、そのクラス$\overline{x+y}$は一意的に決まってしまうことである。実際、別の代表$x^{\prime}\in X,y^{\prime}\in Y$をとったとき
      \begin{align*}
        x+y\equiv x^{\prime}+y^{\prime} \pmod{\a}
      \end{align*}
      であることを見ればよいが、
      \begin{align*}
        (x+y)-(x^{\prime}+y^{\prime})=(x-x^{\prime})+(y-y^{\prime})
      \end{align*}
      で$x-x^{\prime}\in\a,y-y^{\prime}\in\a$であるから$(x-x^{\prime})+(y-y^{\prime})\in\a$となり上の合同式が成立する。

    \item $\g/\a$におけるスカラー倍や\hlin{Lie_bra}{リー積}を
      \begin{align*}
        X=\bar{x}\qc Y=\bar{y}\qq{のとき}\\
        \lambda X=\overline{\lambda x}\qc\comm{X}{Y}=\overline{\comm{x}{y}}
        \tag{11}\label{eq:2_(11)}
      \end{align*}
      によって定義する。
      これが実現可能なことは、上と同様、$x^{\prime}\in X\qc y^{\prime}\in Y$とするとき
      \begin{align*}
        \textcolor{red}{\lambda x-\lambda x^{\prime}=\lambda(x-x^{\prime})\in\a}
      \end{align*}
      \begin{align*}
        &\comm{x}{y}-\comm{x^{\prime}}{y^{\prime}}\\
        =&\comm{x}{y}-\comm{x^{\prime}}{y}+\comm{x^{\prime}}{y}-\comm{x^{\prime}}{y^{\prime}}\\
        =&\comm{x-x^{\prime}}{y}+\comm{x^{\prime}}{y-y^{\prime}}\in\a
      \end{align*}
      であることからわかる。この最後のところで$\a$が(両側){\イデアル}の性質\ref{Def:2_2_ii'}を用いた。
  \end{itemize}
\end{Proof}

\begin{remark}
  上述のように定義された演算は{\Lie環}の条件を満たす。
\end{remark}
\begin{Proof}
  ヤコビの公式\ref{Def:1_3_II}が成立することは、$X,Y,Z\in\g/\a$, $x\in X,y\in Y, z\in Z$とするとき
  \begin{align*}
    \comm{X}{Y}\qq{は}\comm{x}{y}\qq{の{\クラス}}.
  \end{align*}
  したがって
  \begin{align*}
    \comm{\comm{X}{Y}}{Z}\qq{は}\comm{\comm{x}{y}}{z}\qq{のクラス}.
  \end{align*}
  同様にして
  \begin{align*}
    &\comm{\comm{Y}{Z}}{X},\comm{\comm{Z}{X}}{Y}\qq{はそれぞれ}\\
    &\comm{\comm{y}{z}}{x},\comm{\comm{z}{x}}{y}\qq{のクラス}
  \end{align*}
  となり、
  \begin{align*}
    &\comm{\comm{X}{Y}}{Z},\comm{\comm{Y}{Z}}{X},\comm{\comm{Z}{X}}{Y}\qq{は}\\
    &\comm{\comm{x}{y}}{z}+\comm{\comm{y}{z}}{x}+\comm{\comm{z}{x}}{y}(=0)\qq{のクラス}
  \end{align*}
  となる。
\end{Proof}

\begin{definition}[商環]\label{Def:quot_ring}
  同値類の集合$\g/\a$に{\Lie環}の構造を入れたものを\emph{剰余環}または\emph{商環}という。
\end{definition}
※群論において{\イデアル}に対応するのは"正規部分群"で、上記と同様にして"商群"の概念が定義される。

\begin{definition}[標準的準同型]\label{Def:canon_hom}
  もとの{\Lie環}$\g$から\商環$\g/\a$の上への写像
  \begin{align*}
    \varphi\colon \g\ni x\mapsto \bar{x}\in\g/\a \qq{($\bar{x}$は$x$の{\クラス})}
    \tag{12}\label{eq:2_(12)}
  \end{align*}
  が定義でき、\eqref{eq:2_(10)},\eqref{eq:2_(11)}から{\準同型写像}である。この\eqref{eq:2_(12)}の写像を\emph{標準的準同型写像}(canonical homomorphism)という。
\end{definition}

\begin{center}
\begin{tikzpicture}
  \draw[ultra thick] (0,0)--(0,3);
  \draw (-0.1,3)--(0.1,3);
  \draw (-0.1,2)--(0.1,2);
  \draw (-0.1,1)--(0.1,1);
  \draw (-0.1,0)--(0.1,0);
  \draw[ultra thick] (2,1)--(2,3);
  \draw (1.9,3)--(2.1,3);
  \draw (1.9,2)--(2.1,2);
  \draw (1.9,1)--(2.1,1);
  \draw[->] (0.3,3)--node[above]{$\varphi$}(1.7,3);
  \draw[->] (0.3,2)--(1.7,2);
  \draw[->] (0.3,1)--(1.7,1);

  \draw (0,3)node[left]{$\g$};
  \draw (0,2)node[left]{$\h$};
  \draw (0,1)node[left]{$\a$};
  \draw (0,0)node[left]{$\qty{0}$};
  \draw (2,3)node[right]{$\g/\a$};
  \draw (2,2)node[right]{$\h/\a=\varphi(\h)$};
  \draw (2,1)node[right]{$\qty{0}$};
\end{tikzpicture}
\end{center}

\begin{remark}
  今、$\g\supset\h\supset\a$となるような{\部分Lie環}があれば、$\varphi$による$\h$の像$\varphi(\h)=\qty{\bar{x}\mid x\in\h}$は明らかに$\g/\a$の部分リー環で、それ自身{\商環}$\h/\a$と考えられる。

  逆に$\g/\a$の{\部分Lie環}$\h^{\prime}$があれば、その逆像
  \begin{align*}
    \varphi^{-1}(\h^{\prime})=\qty{x\in\g\mid \bar{x}\in\h^{\prime}}
  \end{align*}
  は明らかに$\g$の部分リー環で$\a$を含む。さらにこれらについて
  \begin{align*}
    \varphi^{-1}(\varphi(\h))=\h\qc \varphi(\varphi^{-1}(\h^{\prime}))=\h^{\prime}
  \end{align*}
  となることは容易に確かめられる。よって対応$\h\mapsto\h/\a$により$\g$の$\a$を含む部分リー環と、$\g/\a$の部分リー環とが1対1に対応していることがわかる。特に$\h$が$\g$の{\イデアル}ならば、$\h/\a$も$\g/\a$のイデアルであり、その逆も言える。
\end{remark}

\begin{problem}\label{prob:2}
  上記最後に述べたことを証明せよ。
  \href{sol:2}{(回答)}
\end{problem}



\subsubsection*{Example 6.}

\begin{remark}\htar{rem:der_ideal_comm}{}
  $\a=D\g=\comm{\g}{\g}$とすれば、$\g/\a$において、$X=\bar{x},Y=\bar{y}\in\g/\a$ならば(教科書は$\in\textcolor{red}{\bar{\g}}$と定義されていない記法を用いている)
  \begin{align*}
    \comm{X}{Y}=\comm{\bar{x}}{\bar{y}}=\overline{\comm{x}{y}}=0
    \qq{($\comm{x}{y}\in\a$だから)}
  \end{align*}
  よって$\g/\a$は{\可換なLie環}になる。逆にある{\イデアル}$\a$に対し$\g/\a$が{\可換なLie環}ならば、すべての$x,y\in\g$に対し
  \begin{align*}
    &\overline{\comm{x}{y}}=\comm{\bar{x}}{\bar{y}}=0,\\
    &\therefore \; \comm{x}{y}\in\a
  \end{align*}
  となるから、$D\g\subset\a$である。
  よって、$D\g$は$\g/\a$が可換であるようなイデアル$\a$のうち最小なもの(\textcolor{red}{? \cref{Def:sequence}等は更に小さい{\イデアル}にはならない?})として特徴づけられる。(この場合には$D\g\subset\a\subset\g$となるような任意の{\線形部分空間}$\a$は{\イデアル}になることに注意)
\end{remark}

\begin{theorem}
  一般に{\準同型写像}
  \begin{align*}
    \varphi\colon\g\longrightarrow\g^{\prime}
  \end{align*}
  があるとき、$\a=\hlin{Def:ker}{\Ker}{\varphi}$は$\g$の{\イデアル}であるが(\cref{thm:ker_im})、
  $x\in\g$の$\bmod{\;\a}$の{\クラス}$\bar{x}$と$\varphi(x)$とを対応させることにより$\g/\a$と$\varphi(\g)$の間の1対1対応が得られ、それは明らかに{\Lie環の同型}になる:
  \begin{align*}
    \g/\a\cong\varphi(\g)\tag{13}\label{eq:2_(13)}
  \end{align*}
\end{theorem}
\begin{Proof}
  $x,y\in\g$に対し
  \begin{align*}
    \varphi(x)=\varphi(y) 
    \Leftrightarrow& \varphi(x-y)=0\\
    \Leftrightarrow& x-y\in\a\\
    \Leftrightarrow& x\equiv y \pmod{\a},\\
    &\ie\bar{x}=\bar{y}
  \end{align*}
\end{Proof}

\begin{remark}
  例として、$\g$の{\部分Lie環}$\h$、$\g$の{\イデアル}$\a$、および{\標準的準同型写像}
  \begin{align*}
    \varphi\colon\g\longrightarrow\g/\a
  \end{align*}
  を考える。$\varphi$の$\h$への"制限"
  \begin{align*}
    \varphi\mid\g\colon \h\longrightarrow\g/\a
  \end{align*}
  を考えれば
  \begin{align*}
    \Ker(\varphi\mid\h)=\h\cap\qty(\Ker\varphi)=\h\cap\a
  \end{align*}
  は$\h$の{\イデアル}であり、一方
  \begin{align*}
    \Im\qty(\varphi\mid\h)=\qty{\varphi(x)\mid x\in\h}=\varphi(\h)
  \end{align*}
  は$\g/\a$の{\部分Lie環}であるが(\cref{thm:ker_im})、$\varphi^{-1}(\varphi(x))=x+\a$であるから
  \begin{align*}
    \varphi^{-1}\qty(\varphi(\h))=\bigcup_{x\in\h}(x+\a)=\h+\a
  \end{align*}
  となる。
\end{remark}

\begin{center}
\begin{tikzpicture}
  \draw[ultra thick] (0,1)--(0,3);
  \draw (-0.1,3)--(0.1,3);
  \draw (-0.1,2)--(0.1,2);
  \draw (-0.1,1)--(0.1,1);

  \draw[ultra thick] (1,1)--(1,3);
  \draw (0.9,3)--(1.1,3);
  \draw (0.9,2)--(1.1,2);
  \draw (0.9,1)--(1.1,1);

  \draw[ultra thick] (-1.5,-0.5)--(-1.5,1.5);
  \draw (-1.6,1.5)--(-1.4,1.5);
  \draw (-1.6,0.5)--(-1.4,0.5);
  \draw (-1.6,-0.5)--(-1.4,-0.5);


  \draw[->] (0.3,3)--node[above]{$\varphi$}(0.7,3);
  \draw[->] (0.3,2)--(0.7,2);
  \draw[->] (0.3,1)--(0.7,1);

  \draw (-1.5,1.5)--(0,2);
  \draw (-1.5,0.5)--(0,1);

  \draw[dashed,->] (-1.5,1.5)--node[below]{$\varphi\mid\h$}(0.7,1.9);

  \draw (-1.5,1.5)node[left]{$\h$};
  \draw (-1.5,0.5)node[left]{$\h\cap\a$};
  \draw (-1.5,-0.5)node[left]{$\qty{0}$};
  \draw (0,3)node[left]{$\g$};
  \draw (0,2)node[above left]{$\h+\a$};
  \draw (0,1)node[below left]{$\a$};
  \draw (1,3)node[right]{$\g/\a$};
  \draw (1,2)node[right]{$\varphi(\h)$};
  \draw (1,1)node[right]{$\qty{0}$};
\end{tikzpicture}
\end{center}

\begin{remark}
  上記のように$\h+\a$は$\g$の$\a$を含む{\部分Lie環}で$\varphi(\h)$を$(\h+\a)/\a$と同一視することができる。よって、\eqref{eq:2_(13)}を$\varphi\mid\h$に適用して
  \begin{align*}
    \h/(\h\cap\a)\cong(\h+\a)/\a
  \end{align*}
  という基本的な同型定理が得られる。
\end{remark}

\end{document}