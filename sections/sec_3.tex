\documentclass[../main]{subfiles}
\usepackage{../settings/myexternaldocument}
\setcounter{section}{2}

\begin{document}

\twocolumn[
  \section{可解環、冪零環、エンゲルの定理}\label{sec:3}
  \begin{abstract}
    今回はリー環の中で特に可解環、冪零環と呼ばれるものについて説明する。これは群論における可解群、冪零群に対応する概念であり、リー群とリー環の対応を考えると、実際、可解リー群、冪零リー群はそれぞれ可解リー環、冪零リー環に対応している。
  \end{abstract}
]
\subsection{可解リー環の定義}
\begin{definition}[列]\label{Def:sequence}
  $\g$を{\実数体}$\R$上の{\Lie環}とする。
  \begin{align*}
    \href{Def:der_ideal}{D\g}=\comm{\g}{\g}=\qty{\sum_{i=1}^{m}\comm{x_{i}}{y_{i}}\eval x_{i},y_{i}\in\g, m\ge1}
  \end{align*}
  とおけば、これは$\g$の一つの{\イデアル}になり、{\商環}$\g/D\g$は{\可換なLie環}になる。そこで帰納的に
  \begin{align*}
    D^{k}\g=D\qty(D^{k-1}\g)\qq{}(k=1,2,\cdots)
    \tag{1}\label{eq:3_(1)}
  \end{align*}
  (ただし$D^{0}\g=\g$とおく)と定義すれば
  \begin{align*}
    \g(=D^{0}\g)\supset D\g \supset D^{2}\g \supset \cdots
    \tag{2}\label{eq:3_(2)}
  \end{align*}
  という{\部分Lie環}の\emph{列}(sequence)が得られる。
\end{definition}

\begin{remark}
  $D^{k}\g$の$D^{k-1}\g$に対する関係は$D\g$の$\g$に対する関係と全く同じだから、$D^{k}\g$は$D^{k-1}\g$の{\イデアル}で(\hlin{rem:der_ideal_comm}{Rem.参照})、商$D^{k-1}\g/D^{k}\g$は{\可換なLie環}である。
\end{remark}

\begin{named}\label{nam:seq_length}
  \eqref{eq:3_(2)}の各項の次元を考えれば
  \begin{align*}
    \dim\g \ge \dim D\g \ge \dim D^2\g \ge \cdots
  \end{align*}
  という不等式が得られる。よって$\dim\g=n$とすれば、$n$をこえない自然数が$r$があって
  \begin{align*}
    n=&\dim\g>\dim D\g > \cdots > \dim D^{r}\g\\
    =& \dim D^{r+1}\g
  \end{align*}
  となる。この最後の式から、$D^{r}\g=D^{r+1}\g$。したがって
  \begin{align*}
    D^{r+1}\g = D\qty(D^{r}\g)=D\qty(D^{r+1}\g),\cdots
  \end{align*}
  となるから
  \begin{align*}
    D^{r}\g=D^{r+1}\g = D^{r+2}\g=\cdots
  \end{align*}
  となり、{\列}は$r$番目から先は一定になってしまう。この$r$を列\eqref{eq:3_(2)}の\emph{"長さ"}という。
\end{named}

\begin{definition}\label{Def:sol_Lie_alg}
  もし、ある$k$に対して$D^{k}\g=\qty{0}$となるならば、上記により$D^{r}\g=\qty{0}$であり
  \begin{align*}
    \g\supseteq D\g\supseteq \cdots \supseteq D^{r-1}\g \supseteq D^{r}\g = \qty{0}
    \tag{3}\label{eq:3_(3)}
  \end{align*}
  という{\列}が得られる。この場合、$\g$は$r$個の{\可換なLie環}
  \begin{align*}
    \g/D\g,D\g/D^{2}\g,\cdots,D^{r-1}/D^{r}\g \; D^{r-1}\g
  \end{align*}
  を積み重ねた構造を持っている。このような{\Lie環}を\emph{可解リー環}といい、$D^{k}\g=\qty{0}$となる最小の自然数(上記の$r$)をその"長さ"という。
\end{definition}

\subsubsection*{Example 1.}
\begin{remark}
  {\可換なLie環}($\neq\qty{0}$)は長さ$1$の{\可解Lie環}である。
  また$\g=\qty{0}$ならば、$\g$は長さ$0$の可解リー環とみなされる。
  \begin{Proof}
    $x,y\in\g$とおくと、すべての$D\g\ni\comm{x}{y}$について
    \begin{align*}
      \comm{x}{y}=0  \qq{$\because\g$が{\可換なLie環}}
    \end{align*}
    であるから、長さは$1$となる。
  \end{Proof}
\end{remark}

\subsubsection*{Example 2.}
\begin{theorem}
  $n$次正方行列全体のリー環\hlin{nam:gl}{$\gl_{n}(\R)$}の中で、"上半三角行列"全体の作る部分集合を$\g_{0}$とおく:
  \begin{align*}
    \g_{0}=
    \qty{\left.
      \begin{pNiceMatrix}
        \xi_{11} & \xi_{12} & \cdots & \xi_{1n}\\
        & \xi_{22} & \cdots & \xi_{2n}\\
        \Block{2-2}<\huge>{0}&  & \ddots & \vdots\\
        &  &  & \xi_{nn}
      \end{pNiceMatrix}
      \right| \xi_{ij}\in\R
    }.
    \tag{4}\label{eq:3_(4)}
  \end{align*}
  さらに
  \begin{align*}
    \g_{k}=
    \qty{\left.
        \begin{pNiceMatrix}[small,xdots/shorten=5pt]
            0 & \cdots & 0 & \xi_{1,k+1}  & \cdots  & \xi_{1n}  \\
            & \Ddots & & \Ddots & \Ddots & \Vdots  \\
            & & & & & \xi_{n-k,n}  \\
            \Block{3-3}<\huge>{0} & & & & & 0  \\
            & & & & & \Vdots  \\
            & & & & & 0
            \CodeAfter
              \OverBrace{1-1}{1-3}{k}
        \end{pNiceMatrix}
      \right| \xi_{ij}\in\R
    }
    \tag{5}\label{eq:3_(5)}
  \end{align*}
  とおく。これらが$\gl_{n}(\R)$の{\線形部分空間}であることは明らかだが、加えて(\href{Def:sol_Lie_alg}{可解な}){\部分Lie環}になる。
\end{theorem}
\begin{Proof}
  $\g_{0}$の二つの元
  \begin{align*}
    x=
      \begin{pNiceMatrix}[small]
        \xi_{11} & \xi_{12} & \cdots & \xi_{1n}\\
        & \xi_{22} & \cdots & \xi_{2n}\\
        \Block{2-2}<\large>{0}&  & \ddots & \vdots\\
        &  &  & \xi_{nn}
      \end{pNiceMatrix}\qc
    y=
      \begin{pNiceMatrix}[small]
        \eta_{11} & \eta_{12} & \cdots & \eta_{1n}\\
        & \eta_{22} & \cdots & \eta_{2n}\\
        \Block{2-2}<\large>{0}&  & \ddots & \vdots\\
        &  &  & \eta_{nn}
      \end{pNiceMatrix}
  \end{align*}
  に対して、積$xy$および$yx$を計算してみると、それらはまた上半三角行列で、その対角成分はどちらも
  \begin{align*}
    \xi_{11}\eta_{11}\qc
    \xi_{22}\eta_{22}\qc
    \cdots\qc
    \xi_{nn}\eta_{nn}
  \end{align*}
  となる。よって{\交代積}$\comm{x}{y}=xy-yx$も上半三角行列でその対角成分はすべて$0$となる。
  すなわち、$\comm{x}{y}\in\g_{1}$。したがって
  \begin{align*}
    \comm{\g_{0}}{\g_{0}}\subset\g_{1}.
    \tag{6}\label{eq:3_(6)}
  \end{align*}
  同様に$x\in\g_{k}, y\in\g_{l}$ならば$xy, yx$ともに$\g_{k+l}$に入り、$\comm{x}{y}\in\g_{k+l}$。したがって
  \begin{align*}
    \comm{\g_{k}}{\g_{l}}\subset\g_{k+l}.
    \tag{7}\label{eq:3_(7)}
  \end{align*}
  という関係が得られる。これから$\g_{k}\;{(k=0,1,\cdots)}$が$\g_{0}$の{\イデアル}であることがわかる。

  また
  \begin{align*}
    D\g_{0}\subset\g_{1}\qc D\g_{k}\subset\g_{2k}\qq{($k=1,2,\cdots$)}
  \end{align*}
  が成立するから、$D^{k}\g_{1}\subset\g_{2k}$。したがって、$2^{k}\ge n$となるような$k$をとれば、
  \begin{align*}
    D^{k+1}\g_{0}=D^{k}\g_{1}=\qty{0},
  \end{align*}
  となる。
\end{Proof}


\end{document}
