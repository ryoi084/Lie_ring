\documentclass[../main]{subfiles}
\usepackage{../settings/myexternaldocument}
\setcounter{section}{2}

\begin{document}

\twocolumn[
  \section{可解環、冪零環、エンゲルの定理}\label{sec:3}
  \begin{abstract}
    今回はリー環の中で特に可解環、冪零環と呼ばれるものについて説明する。これは群論における可解群、冪零群に対応する概念であり、リー群とリー環の対応を考えると、実際、可解リー群、冪零リー群はそれぞれ可解リー環、冪零リー環に対応している。
  \end{abstract}
]
\subsection{可解リー環の定義}
\begin{definition}[列]\label{Def:sequence}
  $\g$を{\実数体}$\R$上の{\Lie環}とする。
  \begin{align*}
    \href{Def:der_ideal}{D\g}=\comm{\g}{\g}=\qty{\sum_{i=1}^{m}\comm{x_{i}}{y_{i}}\eval x_{i},y_{i}\in\g, m\ge1}
  \end{align*}
  とおけば、これは$\g$の一つの{\イデアル}になり、{\商環}$\g/D\g$は{\可換なLie環}になる。そこで帰納的に
  \begin{align*}
    D^{k}\g=D\qty(D^{k-1}\g)\qq{}(k=1,2,\cdots)
    \tag{1}\label{eq:3_(1)}
  \end{align*}
  (ただし$D^{0}\g=\g$とおく)と定義すれば
  \begin{align*}
    \g(=D^{0}\g)\supset D\g \supset D^{2}\g \supset \cdots
    \tag{2}\label{eq:3_(2)}
  \end{align*}
  という{\部分Lie環}の\emph{列}(sequence)が得られる。
\end{definition}

\begin{remark}
  $D^{k}\g$の$D^{k-1}\g$に対する関係は$D\g$の$\g$に対する関係と全く同じだから、$D^{k}\g$は$D^{k-1}\g$の{\イデアル}で(\hlin{rem:der_ideal_comm}{Rem.参照})、商$D^{k-1}\g/D^{k}\g$は{\可換なLie環}である。
\end{remark}

\begin{named}\label{nam:seq_length}
  \eqref{eq:3_(2)}の各項の次元を考えれば
  \begin{align*}
    \dim\g \ge \dim D\g \ge \dim D^2\g \ge \cdots
  \end{align*}
  という不等式が得られる。よって$\dim\g=n$とすれば、$n$をこえない自然数が$r$があって
  \begin{align*}
    n=&\dim\g>\dim D\g > \cdots > \dim D^{r}\g\\
    =& \dim D^{r+1}\g
  \end{align*}
  となる。この最後の式から、$D^{r}\g=D^{r+1}\g$。したがって
  \begin{align*}
    D^{r+1}\g = D\qty(D^{r}\g)=D\qty(D^{r+1}\g),\cdots
  \end{align*}
  となるから
  \begin{align*}
    D^{r}\g=D^{r+1}\g = D^{r+2}\g=\cdots
  \end{align*}
  となり、{\列}は$r$番目から先は一定になってしまう。この$r$を列\eqref{eq:3_(2)}の\emph{"長さ"}という。
\end{named}

\begin{definition}\label{Def:sol_Lie_alg}
  もし、ある$k$に対して$D^{k}\g=\qty{0}$となるならば、上記により$D^{r}\g=\qty{0}$であり
  \begin{align*}
    \g\supseteq D\g\supseteq \cdots \supseteq D^{r-1}\g \supseteq D^{r}\g = \qty{0}
    \tag{3}\label{eq:3_(3)}
  \end{align*}
  という{\列}が得られる。この場合、$\g$は$r$個の{\可換なLie環}
  \begin{align*}
    \g/D\g,D\g/D^{2}\g,\cdots,D^{r-1}/D^{r}\g \; D^{r-1}\g
  \end{align*}
  を積み重ねた構造を持っている。このような{\Lie環}を\emph{可解リー環}といい、$D^{k}\g=\qty{0}$となる最小の自然数(上記の$r$)をその"長さ"という。
\end{definition}

\subsubsection*{Example 1.}
\begin{remark}
  {\可換なLie環}($\neq\qty{0}$)は長さ$1$の{\可解Lie環}である。
  ($\g=\qty{0}$ならば、$\g$は長さ$0$の可解リー環とみなされる。)
\end{remark}

\subsubsection*{Example 2.}


\end{document}
