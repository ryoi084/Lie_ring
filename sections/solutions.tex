\documentclass[../main]{subfiles}
\DoIfAndOnlyIfStandAlone{%
  \externaldocument{sec_1}%
  \externaldocument{sec_2}%
  \externaldocument{sec_3}%
  \externaldocument{solutions}%
}%}

\begin{document}
\twocolumn[
  \section*{問題回答}
]

\setcounter{section}{1}
\setcounter{section}{2}
\begin{solution}\label{sol:1}
  \href{prob:1}{(問題文)}\\
  $\g=\gl_n(\R)=M_n(\R)$とする。左辺$D\g$について
  \begin{align*}
    D\g=\comm{\g}{\g} \qq{from \cref{Def:der_ideal}}
  \end{align*}
  $x,y\in\g$において、$\sl_n(\R)$であることを示すには\eqref{eq:sl}より$\tr{\comm{x}{y}}=0$であることを示せばよい。\eqref{eq:sl}に倣って
  $x=(\xi_{ij}),y=(\eta_{ij})$とおくと
  \begin{align*}
    (xy)_{ij}=\sum_{k=1}^{n}\xi_{ik}\eta_{kj}\qc
    (yx)_{ij}=\sum_{k=1}^{n}\eta_{ik}\xi_{kj}
  \end{align*}
  より
  \begin{align*}
    (\comm{x}{y})_{ij}=\sum_{k=1}^{n}\qty(\xi_{ik}\eta_{kj}-\eta_{ik}\xi_{kj})
  \end{align*}
  トレースを取ると
  \begin{align*}
    \tr{\comm{x}{y}}=&\sum_{i=1}^{n}\qty(\comm{x}{y})_{ii}\\
    =&\sum_{i,k=1}^{n}\qty(\xi_{ik}\eta_{ki}-\eta_{ik}\xi_{ki})\\
    =&0
  \end{align*}
  よって$D\g=\sl_n(\R)$である。
\end{solution}

\begin{solution}\label{sol:2}
  \href{prob:2}{(問題文)}
\end{solution}


\end{document}